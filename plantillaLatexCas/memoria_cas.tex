\documentclass[12pt,a4paper,titlepage,twoside]{report}
\usepackage[T1]{fontenc}
\usepackage[english, spanish]{babel}
\usepackage[utf8]{inputenc}
\usepackage{lmodern}
\usepackage{graphicx}
\usepackage{csquotes}
\usepackage{xcolor}
\usepackage{url}
\usepackage[backend=biber,style=alphabetic,
sorting=ynt]{biblatex}
\addbibresource{biblio.bib}
% http://tug.ctan.org/tex-archive/macros/latex/contrib/fancyhdr/
\usepackage{fancyhdr}
\pagestyle{fancy}
\fancyhf{}
\fancyhead[LE,RO]{\nouppercase \rightmark}
\fancyhead[LO,RE]{\nouppercase \leftmark}
\fancyfoot[C]{\thepage}

\newcommand{\Keywords}[1]{\vfill\noindent{\small{\em Palabras clave}: #1}}
\definecolor{grisclar}{gray}{0.5}
\definecolor{grisfosc}{gray}{0.25}

% Editar con los datos correspondientes
\newcommand{\titulo}{Diseño e implementación de un Cluster de virtualización en alta disponibilidad para los servicios de emergencias del Consorcio Provincial de Bomberos de Valencia}
\newcommand{\titulacion}{Ingeniería en Informática}
\newcommand{\autor}{María Alexandra Hermosilla Semikina}
\newcommand{\director}{Alberto Conejero}

\title{\titulo}
\author{\autor}

\begin{document}
\input{portada_cas}

\begin{abstract}
Este proyecto nació a partir de la necesidad de un cluster de virtualización que estuviera en alta disponibilidad para albergar las herramientas que utiliza el Consorcio Provincial de Bomberos de Valencia tanto para los servicios de emergencia como los servicios internos. 
El Consorcio Provincial de Bomberos de Valencia ha ido aumentando los servicios que presta a nivel interno como a nivel operativo en los servicios de emergencia, consecuentemente es necesario ampliar su infraestructura de virtualización como adaptarse al nuevo hardware que se ha incorporado a la infraestructura ya existente.
Para ello, se ha construido un cluster de virtualización basado en tecnologías como Xen, corosync y pacemaker.

\Keywords{Virtualización, HA, Alta disponibilidad, Xen, Servicios de emergencia, Cluster, Debian, Linux,Corosync, Pacemaker, ocfs2, Máquina Virtual,GNU}
\end{abstract}

\tableofcontents

\chapter{Introducción}
En la administración de sistemas siempre ha habido una gran problemática alrededor de la optimización de los recurso. Estos recursos no solo son a nivel de hardware, si no que también a nivel de refrigeración, consumo eléctrico y espacio, entre otros factores. El coste del mantenimiento era extremadamente elevado para entidades que disponían de infraestructuras informáticas muy grandes, ya que no solo era necesario disponer de repuestos, si no también de un elevado número de personal dedicado a esta labor.
\par
Desde la llegada de la virtualización, la informática ha sufrido una gran revolución en este aspecto. La virtualización consiste en una capa de software que corre en un sistema operativo que crea una abstracción entre el hardware y el software a ejecutar. Para este software, esta capa, es totalmente transparente, ya que ve los mismos recursos de hardware que vería un sistema operativo nativo. El software puede ser cualquier recurso que necesitemos como una computadora, sistema operativo, dispositivo de almacenamiento, aplicaciones o redes. Esto permite que, en caso de fallo de hardware, este software se pueda trasladar a otro hardware que corra la misma capa de abstracción y el software no se percate del cambio y siga funcionando sin problemas. 
\par
Como se puede apreciar, la virtualización presente una serie de ventajas respecto a las infraestructuras físicas. De estas ventajas se puede destacar las siguientes:
\begin{itemize}
\item \textbf{Aislamiento:} Cada máquina virtual es independiente una de la otra y del hypervisor. Por lo que si una de ellas falla, no afecta a las demás como tampoco al hypervisor.
\item \textbf{Seguridad:} En caso de lograr acceso privilegiado a una de estas máquinas virtuales, sólo se obtendría acceso a dicha máquina, dejando intactas el resto de máquinas y  hypervisor.
\item \textbf{Flexibilidad:} Las máquinas virtuales, al ser software, se les puede asignar los recursos necesarios (CPU, RAM, HDD...) sin necesidad de comprar un hardware concreto para ellas. Esto permite ampliarlas en un momento de uso alto de recursos o quitarlos cuando no son necesarios.
\item \textbf{Agilidad:} La puesta en marcha de una máquina virtual suele ser un proceso sencillo. Se puede hacer a través de rellenar un fichero con los recursos necesarios o a través de un asistente de creación. Con ello, una máquina virtual está funcionando en cuestión de minutos.
\item \textbf{Portabilidad:} El hecho de que las máquinas virtuales sean un par de ficheros,permite cambiarla de hypervisor o hardware sin gran problema. Solo es mover unos ficheros y volver a ejecutar la máquina.
\end{itemize}
\par 
Todas estas características son las que se busca en una infraestructura donde el número de servicios prestados es alto y han de estar en alta disponibilidad. Este es el caso del Consorcio Provincial de Bomberos de Valencia. Donde el número de servicios de información prestados son elevados y diversos. Algunos de ellos es necesario que estén operativos las 24 del día. 
\par
En octubre de 2012 accedí al Consorcio Provincial de Bomberos de Valencia como becaria para realizar las labores de apoyo en las áreas de administración de sistemas y helpdesk. Los primeros meses de la beca se me formó en las tecnologías usadas en la infraestructura utilizada para dar servicio a la sede central y a los diferentes parques que forman el Consorcio. Posteriormente y ante la urgencia de desplegar nuevos servicios, se me asigno al proyecto de la creación de un nuevo cluster de virtualización sobre el hardware Unified Computing System (UCS) de Cisco con cuatro blades listos para la instalación y configuración del cluster.


\section{Consorcio Provincial de Bomberos de Valencia}
Durante los años 60, gracias a la expansión económica, la provincia de Valencia sufrió una gran transformación socioeconómica que conllevo a un aumento de población e industrialización. Estos factores influyeron de manera notoria en el aumento del riesgo y en la demanda de los servicios de emergencias.Por este motivo, en 1982 se constituyeron los primeros servicios de emergencias en la provincia de Valencia en forma de Consorcios Comarcales. Se agruparon en mancomunidades a varios municipios por comarcas con el objetivo de facilitar la prestación de este servicio. De esta forma, se crearon 7 consorcios comarcales con la ayuda de la Diputación de Valencia: Horta Nord, Horta Sud, Camp de Morvedre, Ribera Baixa, Ribera Alta-Valldigna, La Safor y La Costera.
\par
Esta diseminación dejó patente las carencias y problemas de organización y coordinación. Por este motivo, el 31 de octubre de 1986 se constituye un único órgano provincial, el Consorcio Provincial de Bomberos de Valencia, tras la aprobación de sus Estatutos por la Generalitat, la Diputación de Valencia y los 132 municipios que se integraron en él.
\par
Actualmente, el Consorcio cubre una superficie de 10.671,48 kilómetros cuadrados, 5.819 kilómetros cuadrados de masa forestal, más de 3.000 kilómetros de carreteras, 1.700.829 vehículos y cerca de 184.483 empresas contabilizadas en la provincia de Valencia. Como también el número de pueblos que se encuentran adheridos a esta mancomunidad a ascendido a 265 pueblos. Para realizar su labor de forma eficaz el Consorcio esta formado por 17 parques profesiones, 7 parques voluntarios y una sede central situada en la ciudad de Valencia. La distribución de los parques se agrupa en 6 zonas que cubre la totalidad de la provincia. Cada zona cuenta con un parque principal y al menos un parque auxiliar, dependiendo de la zona a cubrir. Algunas zonas cuentan con parques de voluntarios para completar la cobertura.
\par
En todo momento y a lo largo de toda la provincia se encuentran de guardia más de 100 bomberos que son capaces de atender el 81'14 por ciento de los riesgos que presentan las actividades por Incendios, Salvamentos y Prevenciones por Asistencia Técnica, en menos de 10 minutos.Se atiende el 98,03 por ciento si el tiempo de respuesta lo fijamos hasta los 20 minutos.
El promedio anual de servicios realizados asciende a 14000, pero cada sigue en aumento. Aunque su principal labor consiste en la extinción de incendios urbanos, industriales y forestales, también se realizan rescates de de automovilistas atrapados en accidentes de tráfico que ha ido en aumento a lo largo de los últimos años. 
\section{Servicios de información que se prestan y su taxonomía}
En una entidad como el Consorcio es fácil encontrar una gran variedad de servicios que se presten a nivel interno como también de cara al ciudadano. La mayoría de estos servicios prestados se apoyan en el soporte por parte del departamento de informática quien gestiona las herramientas informáticas usadas.
%REVISAR LOS SERVICIOS CON EL CATALOGO DE SERVICIOS Y BUSCAR UNA CLASIFICACIÓN

\chapter{Antecedentes}
El Consorcio cuenta en estos momentos con un personal que asciende a más de 800 personas, entre las cuales no solo se encuentra el personal operativo, sino también de personal administrativo. Cada uno de estos dos grupos de usuarios requiere de herramientas diferentes y especializadas para su labor. 
\par
Para el personal operativo es necesario que los servicios usados para su labor estén operativos siempre y poder acceder a ellos en cualquier momento y lugar. Por una parte, el Consorcio es el responsable de recibir los servicios solicitados por parte del 112 de la Generalitat Valenciana que requieran la intervención de los bomberos. Sin embargo, esta parte esta gestionada por parte del propio 112 GVA. Para la gestión del servicio a prestar sí se usa un software propio que esta gestionado internamente. Este servicio es usado no solo por parte de los operadores, sino que también por parte de todo el personal operativo. Esto quiere decir que tiene que estar disponible desde cualquier parque de bomberos de la provincia de Valencia, incluyendo los parques de voluntarios.
\par
Por otra parte, el personal administrativo es el que más servicios requiere, ya que son los responsables de toda la gestión de la organización. Esto quiere decir, que a pesar de ser el grupo minoritario, son los que más recursos necesitan para realizar su labor.
\section{Infraestructura previa}
incluir diagramas
\section{Nuevos servicios a prestar}
\section{Requisitos de la solución}

\chapter{Resumen de alternativas y criterios de selección}
\section{Criterios de selección}
\section{Proxmox}
La empresa Proxmox Server Solutions GmbH es un proyecto que se inició en 2005, pero su producto principal, el entorno de virtualización, no empezó hasta el año 2008. Este entorno utiliza dos tecnologías diferentes de virtualización. La primera, enfocada para los sistemas GNU/Linux, basada en OpenVZ. OpenVZ se basa en contenedores que permite ejecutar múltiples instancias del mismo sistema operativo aislados denominados Servidores Privados Virtuales (VPS, siglas en inglés). Cada contenedor tiene sus propias archivos y aplicaciones, usuarios y dispositivos. Todos estos contenedores comparten el mismo kernel
\section{CloudStack}
\section{OpenStack}
\section{VMware vSphere}
\section{Xen + Pacemaker + Corosync}

\chapter{Implementación de la solución seleccionada}
Introducción de cómo se realizo
diagrama de flujo
Separar en pasos y resolución de las problemáticas aparecidas
\chapter{Resultados y conclusiones}
Posibles mejoras
Posibles lineas de investigación
\chapter{Anexos}
Ficheros de configuración

%\chapter{Bibliografía}
\nocite{*}
\printbibliography

\end{document}