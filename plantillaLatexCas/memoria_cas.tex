\documentclass[12pt,a4paper,titlepage,twoside]{report}
\usepackage[T1]{fontenc}
\usepackage[english, spanish]{babel}
\usepackage[utf8]{inputenc}
\usepackage{lmodern}
\usepackage{graphicx}
\usepackage{csquotes}
\usepackage{xcolor}
\usepackage{url}
\usepackage[backend=biber,style=alphabetic,
sorting=ynt]{biblatex}
\addbibresource{biblio.bib}
% http://tug.ctan.org/tex-archive/macros/latex/contrib/fancyhdr/
\usepackage{fancyhdr}
\pagestyle{fancy}
\fancyhf{}
\fancyhead[LE,RO]{\nouppercase \rightmark}
\fancyhead[LO,RE]{\nouppercase \leftmark}
\fancyfoot[C]{\thepage}

\newcommand{\Keywords}[1]{\vfill\noindent{\small{\em Palabras clave}: #1}}
\definecolor{grisclar}{gray}{0.5}
\definecolor{grisfosc}{gray}{0.25}

% Editar con los datos correspondientes
\newcommand{\titulo}{Diseño e implementación de un Cluster de virtualización en alta disponibilidad para los servicios de emergencias del Consorcio Provincial de Bomberos de Valencia}
\newcommand{\titulacion}{Ingeniería en Informática}
\newcommand{\autor}{María Alexandra Hermosilla Semikina}
\newcommand{\director}{Alberto Conejero}

\title{\titulo}
\author{\autor}

\begin{document}
\input{portada_cas}

\begin{abstract}
Este proyecto nació a partir de la necesidad de un cluster de virtualización que estuviera en alta disponibilidad para albergar las herramientas que utiliza el Consorcio Provincial de Bomberos de Valencia tanto para los servicios de emergencia como los servicios internos. 
El Consorcio Provincial de Bomberos de Valencia ha ido aumentando los servicios que presta a nivel interno como a nivel operativo en los servicios de emergencia, consecuentemente es necesario ampliar su infraestructura de virtualización como adaptarse al nuevo hardware que se ha incorporado a la infraestructura ya existente.
Para ello, se ha construido un cluster de virtualización basado en tecnologías como Xen, corosync y pacemaker.

\Keywords{Virtualización, HA, Alta disponibilidad, Xen, Servicios de emergencia, Cluster, Debian, Linux,Corosync, Pacemaker, ocfs2, Máquina Virtual,GNU}
\end{abstract}

\tableofcontents

\chapter{Introducción}
En octubre de 2012 accedí al Consorcio Provincial de Bomberos de Valencia como becaria para realizar las labores de apoyo en las áreas de administración de sistemas y helpdesk. Durante los primeros meses de la beca se me formó en las tecnologías usadas en la infraestructura utilizada para dar servicio a la sede central y a los diferentes parques que forman.

\section{Consorcio Provincial de Bomberos de Valencia}
Durante los años 60, gracias a la expansión económica, la provincia de Valencia sufrió una gran transformación socioeconómica que conllevo a un aumento de población e industrialización.Estos factores influyeron de manera notoria en el aumento del riesgo y en la demanda de los servicios de emergencias.Por este motivo, en 1982 se constituyeron los primeros servicios de emergencias en la provincia de Valencia en forma de Consorcios Comarcales. Se agruparon en mancomunidades a varios municipios por comarcas con el objetivo de facilitar la prestación de este servicio. De esta forma, se crearon 7 consorcios comarcales con la ayuda de la Diputación de Valencia: Horta Nord, Horta Sud, Camp de Morvedre, Ribera Baixa, Ribera Alta-Valldigna, La Safor y La Costera.
Esta diseminación dejó patente las carencias y problemas de organización y coordinación. Por este motivo, el 31 de octubre de 1986 se constituye un único órgano provincial, el Consorcio Provincial de Bomberos de Valencia, tras la aprobación de sus Estatutos por la Generalitat, la Diputación de Valencia y los 132 municipios que se integraron en él.
Actualmente, el Consorcio cubre una superficie de 10.671,48 kilómetros cuadrados, 5.819 kilómetros cuadrados de masa forestal, más de 3.000 kilómetros de carreteras, 1.700.829 vehículos y cerca de 184.483 empresas contabilizadas en la provincia de Valencia. Como también el número de pueblos que se encuentran adheridos a esta mancomunidad a ascendido a 265 pueblos. Para realizar su labor de forma eficaz el Consorcio esta formado por 17 parques profesiones, 5 parques voluntarios y una sede central situada en la ciudad de Valencia. La distribución de los parques se agrupa en 6 zonas que cubre la totalidad de la provincia. 
\section{Servicios de emergencia}
En todo momento y a lo largo de toda la provincia se encuentran de guardia más de 100 bomberos que son capaces de atender el 81'14 por ciento de los riesgos que presentan las actividades por Incendios, Salvamentos y Prevenciones por Asistencia Técnica, en menos de 10 minutos.Se atiende el 98,03 por ciento si el tiempo de respuesta lo fijamos hasta los 20 minutos.
El promedio anual de servicios realizados asciende a 14000, pero cada sigue en aumento. Aunque su principal labor consiste en la extinción de incendios urbanos, industriales y forestales, también se realizan rescates de de automovilistas atrapados en accidentes de tráfico que ha ido en aumento a lo largo de los últimos años. 
\chapter{Antecedentes}
\section{Infraestructura previa}
\section{Servicios prestados}


\chapter{Diseño}

\chapter{Implementación}

\chapter{Resultados y conclusiones}

\chapter{Anexos}

\chapter{Bibliografía}
\nocite{*}
\printbibliography

\end{document}